\section{Introduction}
\par A paper by Lloyd N. Trefethen~\cite{trefethen_definition} offers two definitions of numerical analysis:
\begin{quote}
    \textit{The study of rounding errors.}
\end{quote}
\begin{quote}
    \textit{The study of algorithms for the problems of continuous mathematics.}
\end{quote}

\par Trefethen argues that the first definition, though an accurate summation of the field's history, does not serve to entice the curious to explore the field. He proposes the second as a more compelling definition, one that emphasises the field's role in solving real-world problems, and inspires curiosity.

\par He offers an optimistic view of the field, one that is not bogged down by the minutiae of rounding errors, but rather one that is focused on the algorithms that make numerical analysis possible. More to ground, the majoirity of numerical analysis is concerned with the speed of convergence and minimization of error.

Through my work on this paper, I have honed my own definition:
\begin{quote}
	\textit{Numerical analysis is the field which attempts to straddle the gaps between the discrete and the continuous.}
\end{quote}

In this paper we will take a look at the stability of various numerical methods when applied to problems of differential calculus.\\
In particular, we wish to:
\begin{itemize}
	\item Establish the Exponential Decay Problem as the toy problem for analysis.
	\item Define stability, and its associated concepts, in the context of numerical analysis.
	\item Establish the concept of Complex time steps for numerical methods.
	\item Develop a framework for the analysis of the stability of numerical methods.
	\item Apply this framework to particular numerical methods.
	\item Expand upon the results of this to contrast the stability regions for Real and Complex time steps.
	\item MORE?
\end{itemize}
