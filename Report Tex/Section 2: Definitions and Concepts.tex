\section{Definitions and Concepts}

\subsection{Time Steps}
Let's take a step back to when we first studied derivatives.\\
\term{Newton's Difference Quotient} should be familiar: 
$f'(x) = \Lim{h \to 0} \frac{f(x+h) - f(x)}{h}$

\par Viewing Newton's Difference Quotient in the context of numerical methods, we call $h$ a \term{time step}.\\
Due to computational constraints, we cannot take $h$ to be infinitesimal. (A computer's memory is as small as it is big)\\
Instead, we take $h$ to be a small, finite number.\\
This defines the concept of a \term{step size}.\\

\par In a more general sense, differential equations may be represented as $y'(t) = \phi(t, y(t))$.\\
In this case, $t$ is an independent variable and $y(t)$ is a dependent variable.\\
The function $\phi(t, y(t))$ is the derivative of $y(t)$ with respect to $t$, and is a function of both $t$ and $y(t)$.\\
In the context of numerical methods, we approximate $y'(t)$ by $\frac{y(t+\Delta_t) - y(t)}{\Delta_t}$.\\
Here, $\Delta_t$ is the time step.\\
This gives $y(t + \Delta_t) \approx y(t) + \Delta_t y'(t)$; a first-order approximation of $y$ a small time step $\Delta_t$ in the future.\\
%TODO: WHAT IS THE ERROR EXACTLY?

\subsection{Complex-Variable Method}
Take a function $f: \bR \longrightarrow \bR,\; x \longmapsto f(x)$.\\
Let $f$ be \term{holomorphic} on its domain: $f$ is complex differentiable in a neighbourhood of any $x \in \bR$.\\
$\forall x$, $f$ is complex differentiable on $B_{\beta}(x) = \{z \in \bC \,|\, |z|-|x| < \beta_{_{\text{small}}}\}$ for some $\beta_{_{\text{small}}} > 0$.\\
In this case, we get the following fact:\\
\[f'(x) = \frac{Im\big(f(x+ih)\big)}{h} + O(h^2), \quad \text{where}\; h \in \bR \;\text{and}\; h \neq 0\]\\
%TODO: WHAT IS THE ERROR EXACTLY?

\par Combining this with the first order approximation above, we get $y(t + \Delta_t) \approx y(t) + Im\big(y(t+i\Delta_t)\big)$.

%TODO: Flesh this out more!

\subsection{Stability}
\par In numerical analysis, a method is said to be \term{stable} if small deviations in the input do not lead to large perturbtions in the output.
In the context of differential equations, a method is said to be stable if the solution does not grow to be unbounded as the number of time steps increases.

\par When solving differential equations numerically, the choice of time step is crucial.
If the step size is too large, the solution may become unstable; the numerical solution will diverge from the analytic solution in proportion with the number of steps.
If the step size is too small, the solution may be accurate and stable, but the computation may be too slow.

\par This tradeoff between accuracy and stability is a common theme in numerical analysis, and depends entirely on the choice of numerical method.
