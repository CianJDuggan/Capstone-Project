\section{Complex Time-Steps}
So far, we have restricted ourselves to time-steps in the real domain.\\
It has often proven useful to extend the analysis of a problem to the complex domain.\\
Motivated by a paper by \textit{George, Yung and Mangan}\cite{walking_into_the_complex_domain}, we will have a look at the stability of numerical methods when the chosen time-step is complex.\\
They state: \begin{quote} \textit {Most numerical methods for time integration use real time steps. Complex time steps provide an additional degree of freedom, as we can select the magnitude of the step in both the real and imaginary directions. By time stepping along specific paths in the complex plane, integrators can gain higher orders of accuracy or achieve expanded stability regions.}\end{quote}

\subsection{Complex 2-Step}
\begin{multicols}{2}
\par We want to take an overall step of size $\lh \in \bR$ comprised of two steps $z_1, z_2 \in \bC$ such that $z_1 + z_2 = \lh$.\\
We can set $z_1 = a + bi$ and $z_2 = \lh - a - bi$.\\
For analysing our numerical methods, we require a comparable implementation of the real step setup.\\

\par A real 2-step method where we take a step of size $\frac{\lh}{2}$ allows us to take two steps and go the same distance as the complex 2-step method.\\
We can write this real 2-step method as $y_{j+1} = {s(\lambda, \frac{h}{2})}^{2} y_j$.\\
\textbf{Note:} We can write this explicitly as 
\[y_{j+1} = s(\lambda, \frac{h}{2}) y_{j+\frac{1}{2}} = s(\lambda, \frac{h}{2}) s(\lambda, \frac{h}{2}) y_j\]

\par We can write the complex 2-step method as 
\[y_{j+1} = s(\lambda, z_1) s(\lambda, z_2) y_j\]
\textbf{Note:} We can write this explicitly as 
\[y_{j+1} = s(\lambda, z_1) y_{j+\frac{1}{2}} = s(\lambda, z_1) s(\lambda, z_2) y_j\]

\columnbreak{}
% Triangle diagram
\begin{tikzpicture}
	\draw[->] (0,0) -- (8,0) node[right] {$Re$};
	\draw[->] (0,0) -- (0,8) node[above] {$Im$};
	\draw[-, color=orange] (0,0) -- (5,7) node[midway, shift={(-0.2,0)}, left] {$z_1$};
	\draw[-, color=orange] (5,7) -- (7,0) node[midway, shift={(0.1,0)}, right] {$z_1 + z_2$};
	\draw[color=purple] (0,0) -- (7,0) node[midway, below] {$\lh$};
\end{tikzpicture}
\end{multicols}


\subsection{Complex Conjugate Pairs}
\begin{multicols}{2}
% Triangle diagram
\begin{tikzpicture}
	\draw[->] (0,0) -- (7,0) node[right] {$Re$};
	\draw[->] (0,0) -- (0,4.5) node[above] {$Im$};
	\draw[-, color=orange] (0,0) -- (3,3.5) node[midway, shift={(-0.2,0)}, left] {$z$};
	\draw[-, color=orange] (3,3.5) -- (6,0) node[midway, shift={(0.1,0)}, right] {$z + \bar{z}$};
	\draw[-, color=purple] (0,0) -- (3,0) node[midway, below] {$\frac{\lh}{2}$};
	\draw[-, color=purple] (3,0) -- (6,0) node[midway, below] {$\frac{\lh}{2}$};
	\draw[-, color=blue] (3,0) -- (3,3.5) node[midway, right] {$b$};
\end{tikzpicture}
\columnbreak{}
\par When we restrict the exponential decay problem by $\lambda \in \bR$, we also find that $s(\lambda, z_1) s(\lambda, z_2) \in \bR$\\
This is because any value attained by $e^{\lambda t}$ is real.\\
This restricts the values of $z_1$ and $z_2$, as any imaginary terms in $s(\lambda, z_1) s(\lambda, z_2)$ must cancel out.\\

\par One particularly simple family of such complex pairs is defined by setting $a = \frac{\lh}{2}$\\
This necessitates $z_2 = \bar{z_1}$\\
We will refer to this family as the \term{complex conjugate step pairs}.\\
\end{multicols}

\newpage
\subsection{Stability of 2-Step Methods for Complex Conjugate Step Pairs}
\par In this section we will compare the stability regions for the real and complex 2-step methods.\\
The real step size is given by $\frac{\lh}{2}$\\
The complex conjugate step pair is given by $z = \frac{\lh}{2} + bi$ and $\bar{z} = \frac{\lh}{2} - bi$\\
The plots in this section show the regions of stability $S = \{h : |s(\lambda, h)| \leq 1\}$ for the case where $b=a=\frac{\lh}{2}$\\
We will have a look at varying $b$ in a subsequent section.\\

\subsubsection{Euler's Forward 2-Step}
\begin{multicols}{2}
\begin{center}
\includegraphics[width=0.49\textwidth]{Stability Regions/Graphs/Real VS Complex Comparison/Euler's Forward.png}
\end{center}
\columnbreak{}

\textbf{Euler's Forward 2-Step $\bR$}
\begin{flalign*}
	y_{j+1} &= {\Big(1+ \frac{\lh}{2}\Big)}^2 y_j && \\
	\implies &s_{_{\bR}}(\lambda, h) = 1 + \lh + \frac{{(\lh)}^2}{4} &&
\end{flalign*}

\textbf{Euler's Forward 2-Step $\bC$}
\begin{flalign*}
	y_{j+1} &= \Big(1 + z\Big)\Big(1 + \bar{z}\Big) y_j && \\
	    &= \Big(1 + \frac{\lambda h}{2} + bi\Big)\Big(1 + \frac{\lambda h}{2} - bi\Big) y_j && \\
	    &= \bigg(\Big{(1 + \frac{\lambda h}{2}\Big)}^2 + b^2\bigg) y_j && \\
    \implies &s_{_{\mathbb{C}}}(\lambda, h) = 1 + \lambda h + \frac{{(\lambda h)}^2}{4} + b^2 && \\
    \implies &s_{_{\mathbb{C}}}(\lambda, h) = s_{_{\mathbb{R}}}(\lambda, h) + b^2 &&
\end{flalign*}

\vspace*{\fill}
\end{multicols}

\par As $b \in \bR$, $s_{_{\mathbb{C}}}(\lambda, h) \geq s_{_{\mathbb{R}}}(\lambda, h)$\\
This means the stability region for the complex 2-step method is smaller than that of the real 2-step method; there are less values of $\lh$ for which the complex 2-step method is stable.\\
$S_{_{\mathbb{C}}}$ is smaller than $S_{_{\mathbb{R}}}$\\
The diagram above illustrates the case where $b = a = \frac{\lh}{2}$\\

\par We can vary the $b$ value to see how the stability region changes.\\
This series of diagrams shows the stability regions for the complex 2-step method with varying $b$ values.\\

%VIDEOS SHOW b BEING VARIED\\

\par We can do the same kind of analysis for other Numerical Methods.\\
Below are the same derivations and diagrams for the Backward Euler and Runge-Kutta 4 methods.\\
\newpage

\subsubsection{Euler's Backward 2-Step}
\begin{multicols}{2}
\begin{center}
\includegraphics[width=0.49\textwidth]{Stability Regions/Graphs/Real VS Complex Comparison/Euler's Backward.png}
\end{center}
\columnbreak{}

\textbf{Euler's Backward 2-Step $\bR$}
\begin{flalign*}
	y_{j+1} &= {\bigg(\frac{1}{1-\frac{\lh}{2}}\bigg)}^2 y_j && \\
	\implies & s_{\bR}(\lambda, h) = \frac{1}{1 - \lh + \frac{{(\lh)}^2}{4}} &&
\end{flalign*}

\textbf{Euler's Backward 2-Step $\bC$}
\begin{flalign*}
	y_{j+1} &= \bigg(\frac{1}{1-\frac{\lh}{2}+bi}\bigg)\bigg(\frac{1}{1-\frac{\lh}{2}-bi}\bigg)y_j && \\
    \implies &s_{\bC}(\lambda, h) = \frac{1}{1 - \lh + \frac{{(\lh)}^2}{4} + b^2} && \\
\end{flalign*}

\vspace*{\fill}
\end{multicols}

\par Again, $b \in \bR$, so this time, $s_{_{\mathbb{C}}}(\lambda, h) \leq s_{_{\mathbb{R}}}(\lambda, h) \implies S_{_{\mathbb{C}}}$ is larger than $S_{_{\mathbb{R}}}$\\

\subsubsection{Runge-Kutta 4 2-Step}
\begin{multicols}{2}
\begin{center}
\includegraphics[width=0.49\textwidth]{Stability Regions/Graphs/Real VS Complex Comparison/Runge-Kutta 4.png}
\end{center}
\columnbreak{}

\textbf{Runge-Kutta 4 2-Step $\bR$}
\begin{flalign*}
    y_{j+1} &= {\bigg(1+z+\frac{z^2}{2}+\frac{z^3}{6}+\frac{z^4}{24}\bigg)}^2 y_j && \\
    y_{j+1} &= \bigg(1+\lh+{(\lh)}^2+\frac{{(\lh)}^3}{6}+\frac{{(\lh)}^4}{24} && \\
            &\quad\,\,\,\,+\frac{{(\lh)}^5}{128}+\frac{5{(\lh)}^6}{4608}+\frac{{(\lh)}^7}{9216}+\frac{{(\lh)}^8}{147456}\bigg) y_j && \\
    \implies &s_{_{\bR}}(\lambda, h) = 1+\lh+{(\lh)}^2+\frac{{(\lh)}^3}{6}+\frac{{(\lh)}^4}{24} && \\
            &\quad\,\,\,\,+\frac{{(\lh)}^5}{128}+\frac{5{(\lh)}^6}{4608}+\frac{{(\lh)}^7}{9216}+\frac{{(\lh)}^8}{147456} &&
\end{flalign*}

\end{multicols}

Computer Calculated Complex; should probably be checked\\
$1 + \lh + \frac{\lh^2}{2} + \frac{\lh^3}{6} + \frac{\lh^4}{24} + \frac{\lh^5}{128} + \frac{5 \lh^6}{4608} + \frac{\lh^7}{9216} + \frac{\lh^8}{147456} + \frac{b^2 \lh^3}{48} + \frac{b^2 \lh^4}{128} + \frac{b^2 \lh^5}{768} + \frac{b^2 \lh^6}{9216} + \frac{b^4 \lh}{24} - \frac{b^4 \lh^2}{96} + \frac{b^4 \lh^3}{192} + \frac{b^4 \lh^4}{1536} - \frac{b^6}{72} + \frac{b^6 \lh}{144} + \frac{b^6 \lh^2}{576} + \frac{b^8}{576}$
