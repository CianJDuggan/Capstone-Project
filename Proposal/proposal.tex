\documentclass[a4paper, fleqn]{article}
\usepackage[margin=0.5in]{geometry} % page layout
\usepackage{amsmath,amssymb,amsthm} % maths packages
\usepackage{tikz-cd} % commutative diagrams
\usepackage[colorlinks, linkcolor=black, urlcolor=black, citecolor=black]{hyperref} % hyperreferences with black links
\usepackage{graphicx} % graphics
\usepackage{multicol} % multiple columns
\usepackage[none]{hyphenat}  % Prevent hyphenation
\usepackage{microtype} % Improves spacing

\setlength{\parindent}{5pt} % Set the indent for new paragraphs to 0
\setlength{\mathindent}{0pt} % Set the indent for equations to 0

\tolerance=1000
\emergencystretch=\maxdimen
\hyphenpenalty=1000
\hbadness=10000

% Define the custom \term command for highlighting terms
\newcommand{\term}[1]{\textcolor{blue}{\textit{#1}}} % blue italics
\newcommand{\Lim}[1]{\raisebox{0.5ex}{\scalebox{0.8}{$\displaystyle \lim_{#1}\;$}}}

\begin{document}
\title{\bf Capstone Project Proposal}
\author{Cian J. Duggan}
\maketitle
\par An issue arises when we attempt to apply computational methods to solve problems in calculus;
Computation is a discrete exercise, while calculus is continuous.
Bridging the gaps is the field of numerical analysis.\\

\par Core to numerical analysis are numerical methods, the algorithms by which we approximate solutions to problems in calculus.
\term{Error}, regarded simply, is the difference between the true solution and the approximation.\\

\par Through one perspective, numerical analysis is the effort to minimize this error.
Another perspective might lend greater weight to maximising the stability of methods.
\term{Stability} is the property of a method that ensures that small deviations in an input do not lead to large perturbations in an output.
In truth, the two concepts are inherintly intertwined, and the balance between them is a central concern in numerical analysis.\\

\par The proposed paper will focus on the stability of numerical methods for solving differential equations.
Differential equations are equations that relate a function to its derivatives.
They are used to model a wide range of phenomena in science and engineering.
Roughly described, a numerical method for solving a differential equation replaces the infitesimal steps of calculus with discrete steps.
These discrete steps are refered to as \term{time steps}.
A paper by George, Jung \& Mangan~\cite{walking_into_the_complex_domain} implicates the utility of taking complex steps in minimising error and takes a look at stability therein.\\

\par In the context of differential equations, a numerical method is said to be stable if the solution does not grow to be unbounded as the number of time steps increases.
The stability of a method will depend on the properties of the differential equation being solved.
Thus, for a specific problem, a set of stable steps can be determined.
This set is commonly referred to as the \term{stability region}.\\

\par The proposed paper seeks to explore the properties of stability regions for various numerical methods, when applied to the exponential decay problem.
Of greatest interest are the properties of stability regions when the chosen time step is complex.\\

\par The exponential decay problem is a simple differential equation that models the decay of a quantity over time.
It is defined by the equation $y'(t) = \lambda y(t)$, where $y$ is the quantity and $\lambda$ is a negative constant.
The solution to this equation is $y(t) = y_0 e^{\lambda t}$, where $y_0$ is the initial quantity.
This problem is often used as a test case for numerical methods, for a plethora of reasons.\\

\par The proposed paper should aim to do the following:
\begin{itemize}
    \item Introduce the exponential decay problem in greater detail.
    \item Define and discuss stability, stability equations, and stability regions.
    \item Explain the use of complex time steps in the setting of numerical methods.
    \item Build a framework for analysing the stability regions of numerical methods for the exponential decay problem when the time step is complex. This framework will include:
    \begin{itemize}
	\item The means by which to determine the stability region for a given numerical method.
	\item The means by which to compare stability regions for real and complex time steps.
	\item The means to explore variance in stability regions for choice of complex time steps.
    \end{itemize}
    \item Apply this framework to specific, simple numerical methods.
    \item Analyse the output of these methods and compare their stability regions.
    \item Discuss the properties of the stability regions and the implications of complex time steps.
    \item Exhibit graphical and video representations of the stability regions for the chosen methods.
Discuss the properties of the stability regions and the implications of complex time steps.
Exhibit graphical and video representations of the stability regions for the chosen methods.
\end{itemize}

\par The proposed paper should be a combination of theoretical and computational work.
The theoretical work will involve the development of the framework and the analysis of stability equations.
The computational work will involve the implementation of the numerical methods and the exploration of the resulting stability regions for different time steps.\\

\par An additional personal goal for this project is utilising GitHub~\cite{GitHub_Repo} for its management.
This is a skill I have been meaning to develop, and I believe the project will provide a good opportunity to do so.\\


\providecommand{\bysame}{\leavevmode\hbox to3em{\hrulefill}\thinspace}
\providecommand{\href}[2]{#2}
\begin{thebibliography}{1}

\bibitem{walking_into_the_complex_domain}
Jithin D. George, Samuel Y. Jung, and Niall M. Mangan, \emph{{W}alking into the complex plane to `order' better time integrators}, \url{https://arxiv.org/abs/2110.04402}, 2021.

\bibitem{GitHub_Repo}
The GitHub repository for this project, \url{https://github.com/CianJDuggan/Capstone-Project}

\end{thebibliography}

\end{document}
